\chapter{Исследовательская часть}
В текущем разделе будут представлены пример работы разработанного программного обеспечения, постановка эксперимента и сравнительный анализ реализованных алгоритмов.

\section{Пример работы программного обеспечения}

На рисунке \ref{img:5.PNG} представлен результат работы программы, на вход подавалась матрица смежности графа, состоящего из 4 вершин. Результат был получен для алгоритма полного перебора и муравьиного алгоритма в виде длины кратчайшего пути и порядка посещения вершин. В качестве эталонного результата считается результат алгоритма полного перебора, ошибка показывает отклонение результа муравьиного алгоритма от алгоритма полного перебора.

\img{110mm}{5.PNG}{Пример работы программы}
\clearpage

\section{Технические характеристики}

Технические характеристики устройства, на котором выполнялась параметризация муравьиного алгоритма решения задачи коммивояжера:

\begin{itemize}
	\item операционная система --- Windows 10~\cite{windows10};
	\item оперативная память --- 16 Гб;
	\item процессор --- Intel® Core™ i5 10300H 2.5 ГГц;
	\item 4 физических ядра, 4 логических ядра.
\end{itemize}

Во время выполнения параметризации алгоритма ноутбук был включен в сеть питания и нагружен только встроенными приложениями окружения и системой тестирования.

\section{Постановка эксперимента}

Автоматическая параметризация была проведена на трех классах данных --- \ref{par:class1} и \ref{par:class2}. Алгоритм будет запущен для набора значений $\alpha, \rho \in (0.1, 0.25, 0.5, 0.75, 0.9), t_{max} \in (20, 50, 100, 200, 500)$.

Итоговая таблица значений параметризации будет состоять из следующих столбцов:
\begin{itemize}[label=---]
	\item $\alpha$ --- коэффициент жадности;
	\item $\rho$ --- коэффициент испарения;
	\item $t_{max}$ --- время жизни колонии муравьёв;
	\item граф 1 --- отклонение результата муравьиного алгоритма от эталонного (полученного алгоритмом полного перебора) на 1-ом классе данных;
        \item граф 2 --- отклонение результата муравьиного алгоритма от эталонного (полученного алгоритмом полного перебора) на 2-ом классе данных;
        \item граф 3 --- отклонение результата муравьиного алгоритма от эталонного (полученного алгоритмом полного перебора) на 3-ем классе данных;
	\item медиана --- медианное значение ошибки муравьиного алгоритма для всех трех классов данных.
\end{itemize}

\textit{Цель эксперимента} --- определить комбинацию параметров, которые позволяют решать задачу коммивояжера наилучшим образом для выбранного класса данных.

Полная таблица параметризации представлена в <<Приложении А>>.

\subsection{Класс данных 1}
\label{par:class1}

Класс данных 1 представляет собой матрицу смежности с линейной размерностью 5 (небольшой разброс значений --- от 1 до 2).

При проведении эксперимента с классами данных было получено, что на первом классе данных муравьиный алгоритм лучше всего показывает себя при параметрах:
\begin{itemize}[label=---]
	\item $\alpha = 0.1,  \rho = 0.1, 0.5, 0.75, 0.9$;
	\item $\alpha = 0.25,  \rho = 0.25, 0.9$;
	\item $\alpha = 0.5,  \rho = 0.1, 0.75, 0.9$;
	\item $\alpha = 0.75,  \rho = 0.5$;
        \item $\alpha = 0.9,  \rho = 0.1, 0.25, 0.9$.
\end{itemize}  
Следовательно, для первого класса данных рекомендуется использовать данные параметры. 

%Для данного класса данных приведена таблица \ref{tbl:table_kd1}	с выборкой параметров, которые наилучшим образом решают поставленную задачу, полные результаты параметризации приведены в приложении А. Использованы следующие обозначения: Days --- количество дней, Result --- результат работы, Mistake --- ошибка как отклонение решения от эталонного .


\subsection{Класс данных 2}\label{par:class2}
Класс данных 2 представляет собой матрицу смежности с линейной размерностью 9 (маленький разброс значений --- от 9 до 11).

Для класса данных 2 было получено, что наилучшим образом алгоритм работает на значениях параметров, которые представлены далее:
\begin{itemize}[label=---]
	\item $\alpha = 0.1, \rho = 0.75$;
	\item $\alpha = 0.25, \rho = 0.1$;
	\item $\alpha = 0.5, \rho = 0.1, 0.25, 0.5, 0.9$;
	\item $\alpha = 0.75, \rho = 0.1, 0.25, 0.5, 0.9$;
	\item $\alpha = 0.9, \rho = 0.5$.
\end{itemize} 
Следовательно, для второго класса данных рекомендуется использовать данные параметры.

\subsection{Класс данных 3}
Класс данных 3 представляет собой матрицу смежности с линейной размерностью 9 (большой разброс значений --- от 0 до 1000).

Для класса данных 3 было получено, что наилучшим образом алгоритм работает на значениях параметров, которые представлены далее:
\begin{itemize}[label=---]
	\item $\alpha = 0.1, \rho = 0.1, 0.25, 0.5, 0.75$;
	\item $\alpha = 0.25, \rho = 0.25$;
	\item $\alpha = 0.5, \rho = 0.1, 0.5, 0.75$;
	\item $\alpha = 0.75, \rho = 0.5, 0.75$;
	\item $\alpha = 0.9, \rho = 0.1, 0.9$.
\end{itemize} 
Следовательно, для третьего класса данных рекомендуется использовать данные параметры.
\section{Вывод}

%В результате эксперимента было получно, что использование муравьиного алгоритма наиболее эффективно при больших размерах матриц. Так, при размере матрицы, равном 2, муравьиный алгоритм медленее алгоритма полного перебора в 153 раза, а при размере матрицы, равном 9, муравьиный алгоритм быстрее алгоритма полного перебора в раз, а при размере в 10 -- уже в 21 раз. Следовательно, при рамзерах матриц больше 8 следует использовать муравьиный алгоритм, но стоит учитывать, что он не гарантирует получения глобального оптимума при решении задачи.
В данном разделе будут представлены пример работы разработанного программного обеспечения, постановка эксперимента и сравнительный анализ отклонения решений задачи коммивояжера реализованными алгоритмами.
 В рамках эксперимента были получены рекомендованные параметры для каждого из трех классов данных, также было выяснено, что для всех классов данных время жизни муравьиной колонии (число дней) значительно влияет на точность решения: чем оно больше, тем меньше отклонение результата работы реализации муравьиного алгоритма от эталонного решения, полученного полным перебором.
