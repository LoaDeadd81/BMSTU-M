\section*{Задание}

В информационный центр приходят клиенты через интервалы времени $10\pm2$~минуты.
Если все 3 имеющихся оператора заняты, клиенту отказывают в обслуживании.
Операторы имеют разную производительность и могут обеспечивать обслуживание
запросов за $20\pm5$,~$40\pm10$,~$40\pm20$~минут. Клиенты стараются
занять свободного оператора с максимальной производительностью.

Полученные запросы сдаются в накопители, откуда они выбираются на
обработку. На первый компьютер поступают запросы от первого и второго операторов, на
второй компьютер --- от третьего оператора. Время обработки на первом и втором
компьютере равно 15 и 30 минутам соответственно.

С помощью языка моделирования GPSS смоделировать процесс обработки 300 запросов. Определить вероятность отказа.а.

\section*{Теоретические сведения}
\textbf{Концептуальная модель}

На рисунке \ref{img:concept.png} представлена концептуальная модель задачи.
\imgw{160mm}{concept.png}{Концептуальная модель}

\textbf{Схема модели СМО}

На рисунке \ref{img:model.png} представлена схема модели информационного центра (как системы массового обслуживания).
\imgw{160mm}{model.png}{Схема модели информационного центра (СМО)}

В процессе взаимодействия клиентов с информационным центром возможно:
\begin{enumerate}
    \item Режим нормального обслуживания, т.е. клиент выбирает одного из свободных
операторов (по заданию клиент пытается занять свободного оператора с максимальной производительностью).
    \item Режим отказа в обслуживании клиента, когда все операторы заняты.
\end{enumerate}

\textbf{Переменные и уравнения имитационной модели}

Эндогенные переменные: время обработки задания i-ым оператором, время решения этого задания j-ым компьютером.

Экзогенные переменные: число обслуженных клиентов ($N_0$) и число клиентов получивших отказ ($N_1$). 

Уравнение (вероятность отказа в обслуживании): 
$P = \frac{N_1}{(N_0 + N_1)}$

\section*{Результаты работы программы}
На рисунке 3 представлен результат работы программы (вероятность
отказа) для исходных параметров, которые указаны в задании. В результате моделирования работы информационного центра было получены следующие результаты: количество обработанных заявок — 231, количество заявок, получивших отказ в обслуживании — 69, вероятность отказа — 23\%.

\imgw{160mm}{demo.png}{Результат работы программы}

