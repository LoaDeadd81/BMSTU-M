\section*{Марковские процессы}
Случайный процесс называется марковским, если для каждого момента времени вероятность любого состояния системы в будущем зависит только от состояния системы в настоящем и не зависит от того, когда и как система пришла в это состояние. 

\section*{Cреднее относительное время пребывания\\системы}

Предельная вероятность состояния показывает среднее относительное время пребывания системы в этом состоянии. Для их поиска используются уравнения Колмогорова, которые имею вид:

\begin{equation}
	P_i'(t) = \sum\limits_{j=1}^{n} \lambda_{ji}P_j(t) - P_i(t)
	\sum\limits_{j=1}^{m} \lambda_{ij},
\end{equation}

\noindent где $P_i(t)$ -- вероятность того, что система находится в $i$-ом состоянии; $n$ --- число состояний в системе из которых можно перейти в $i$--е состояние; $ь$ --- число состояний в системе в которые можно перейти из $i$--го состояния; $\lambda_{ij}$ --- интенсивность перехода системы из $i$-ого состояния в $j$-ое.
Одно из уравнений данной системы заменяется условием нормировки:

\begin{equation}
	\sum\limits_{i=1}^{n} P_i(t) = 1.
\end{equation}

В силу того, что предельные вероятности состояний постоянны, для их определения в уравнениях Колмогорова необходимо заменить их производные нулями и решить полученную систему линейных алгебраических уравнений.

\clearpage

\section*{Точки стабилизации}

Для определения точек стабилизации системы определяются вероятности состояний с
некоторым малым шагом $\Delta t$. Точка стабилизации считается найденной, если
приращение вероятности, а также разница между ранее найденной предельной
вероятностью состояния и вычисленной вероятностью, достаточно малы, то есть
выполняются соотношения:

\begin{equation}
	|P_i(t + \Delta t) -  P_i(t)| < \varepsilon,
\end{equation}

\begin{equation}
	|P_i(t) -  \lim_{t \rightarrow \infty} P_i(t)| < \varepsilon,
\end{equation}

где $\varepsilon$ --- заданная точность.