\section*{Задание}

Разработать программу, которая генерирует последовательность псевдослучайных одно-, двух-, трехразярдных целых чисел. Обеспечить возможность ввода 10 чисел.
Разработать критерий оценки случайности последовательности чисел. Для каждой последовательности вывести число --- случайность данной последовательности.


\section*{Теоретические сведения}

\textbf{Способы генерации случайных чисел}:
\begin{itemize}
    \item Аппаратный способ;
    \item Табличный способ;
    \item Алгоритмический способ.
\end{itemize}


\textbf{Алгоритмический способ}

Для получения случайных чисел алгоритмическим способом выбран вихрь Мерсенна.

Вихрь Мерсенна --- генератор псевдослучайных чисел (ГПСЧ), алгоритм, разработанный в 1997 году японскими учёными Макото Мацумото и Такудзи Нисимура. Вихрь Мерсенна генерирует псевдослучайные последовательности чисел с периодом равным одному из простых чисел Мерсенна, отсюда этот алгоритм и получил своё название и обеспечивает быструю генерацию высококачественных по критерию случайности псевдослучайных чисел. 

Числом Мерсенна называется натуральное число $M_n = 2^n - 1$.

Существуют несколько вариантов алгоритма Мерсенна, различающихся только величиной используемого простого числа Мерсенна. В данной работе будет использован алгоритм MT19937. 

Этапы алгоритма вихрь Мерсенна: 

Шаг 0. Предварительно инициализируется значение констант u1, h1, a:

  u1 ← 10…0   битовая маска старших w-r бит,
  
  h1 ← 01…1   битовая маска младших r бит,
  
  a ← aw-1aw-2…a0  последняя строка матрицы A.

 Шаг 1. x[0], x[1],…,x[n-1] ←  начальное    заполнение

 Шаг 2. Вычисление (xiu | xi+1l)
 
  y ← (x[i] AND u1) OR (x[(i + 1) mod n] AND h1)

 Шаг 3. Вычисляется значение следующего элемента последовательности по 
 рекуррентному выражению
 
  x[i] ← x[(i + m) mod n] XOR (y>>1) XOR a    если младший бит y = 1

  или
  
  x[i] ← x[(i + m) mod n] XOR (y>>1) XOR 0   если  младший бит y = 0

 Шаг 4. Вычисление x[i]T
 
  y ← x[i]
  y ← y XOR (y>>u)
  y ← y XOR ((y<<s) AND b)
  y ← y XOR ((y<<t) AND c)
  z ← y XOR (y>>l)
  вывод z
  
 Шаг 5. i ← (i + 1) mod n 

 Шаг 6. Перейти к шагу 2.

Алгоритм работы вихря Мерсенна состоит из двух частей: рекурсивной генерации и закалки. Рекурсивная часть представляет собой регистр сдвига с линейной обратной связью, в котором все биты слова определяются рекурсивно.

Регистр сдвига состоит из 624 элементов (19937 клеток). Первый элемент состоит из 1 бита, а остальные -- из 32. Процесс генерации начинается с логического умножения на битовую маску, которая отбрасывает 31 бит (кроме наиболее значащих). Следующим шагом выполняется инициализация (x0, x1,…, x623) любыми беззнаковыми 32-разрядными целыми числами. Следующие шаги включают в себя объединение и переходные состояния.

\textbf{Табличный способ}

В качестве таблицы для генерации случайных чисел табличным способом используются
цифры из части таблицы <<A~Million Random Digits with~100,000~Normal~Deviates>> (1955 год).

Данная таблица сохранена в виде текстового файла, где все цифры перечислены без пробелов. Для генерации чисел выбирается
начальная позиция в файле, читаются следущие $n$~цифр, где $n$ --- количество
разрядов в генерируемом числе. Для генерации следующего числа происходит переход к следующей строке
таблицы с сохранением номера столбца. При невозможности перейти к следующей строке в связи с окончанием файла позиция переводится на первую строку, а номер столбца увеличивается на единицу. 
В случае, если в строке не хватает для формирования числа, то они берутся из начала следующей строки файла.

\textbf{Критерий оценки случайности последовательности}

В качестве критерия случайности было использовано отношение средних
арифметических четных и нечетных элементов друг к другу. Чем ближе
коэффициент к 1, тем последовательность более случайна. 
