\section*{Задание}

С помощью языка моделирования GPSS разработать программу, которая предоставляет возможность моделирования работы системы, состоящей из генератора сообщений (выдает сообщения по равномерному закону), буферной памяти (работающей по принципу FIFO), обслуживающего
аппарата (обрабатывает сообщения по закону Пуассона — из ЛР1). С определенной вероятностью часть обработанных сообщений снова поступает в
очередь. Найти минимальный объем очереди (размер буферной памяти),
при котором сообщения не будут теряться.

\section*{Теоретические сведения}
\textbf{Используемые законы распределения}

\textbf{Равномерное распределение}

Функция плотности распределения $f(x)$ случайной величины $X$, имеющей равномерное распределение на отрезке $[a, b]$ ($X \sim R(a, b)$), где $a, b \in R$, имеет следующий вид:
\begin{equation}
	f(x)=\begin{cases}
		\frac{1}{b - a}, & x \in [a, b] \\
		0, & \text{иначе}.
	\end{cases}
\end{equation}

Соответствующая функция распределения $F(x) = \int_{-\infty}^{x}f(t)dt$ принимает вид: 
\begin{equation}
	F(x)=\begin{cases}
		0, & x < a \\
		\frac{x - a}{b - a}, & x \in [a, b] \\
		1, & x > b.
	\end{cases}
\end{equation}

Момент времени $t_i$ может быть вычислен по следующей формуле:

\begin{equation}
	t_i = a + (b - a) R, 
\end{equation}

где $R \in [0, 1]$ --- равномерно распределенная случайная величина в
промежутке~$[0, 1]$.

\textbf{Распределение Пуассона}

Дискретная случайная величина $X$ имеет закон распределения Пуассона с параметром $\lambda$ ($X \sim \Pi(\lambda)$), где $\lambda > 0$, если она принимает значения $0, 1, 2,...$ с вероятностями:

\begin{equation}
	P(X = k)= e^{-\lambda}\frac{\lambda^{k}}{k!}, \quad k \in \{0, 1, 2, ...\}
\end{equation}

Соответствующая функция распределения принимает вид:

\begin{equation}
	F(x) = e^{-\lambda}\sum_{k=0}^{x-1}\frac{\lambda^{k}}{k!} 
\end{equation}

Для генерации Пуассоновских переменных можно использовать метод точек, в основе
которого лежит генерируемое случайное значение $R_i$, равномерно распределенное на
[0, 1], до тех пор, пока не станет справедливым:
\begin{equation}
	\sum_{i=0}^{x}{R_i} >= e^{-\lambda} > \sum_{i=0}^{x + 1}{R_i}
\end{equation}

\textbf{Язык GPSS}

Язык GPSS — общецелевая система моделирования. Как и любой
язык программирования, она содержит словари и грамматику, с помощью
которых разрабатываются точные модели системы (системы массового обслуживания).

\section*{Результаты работы программы}

На рисунках \ref{img:0.png}--\ref{img:99.png} представлены результаты работы программы. С
увеличением вероятности повторной обработки сообщений максимальный
размер очереди (столбец MAX\_CONT.) растет. Получены следующие значения максимальной длины очереди в зависимости от заданной вероятности повторной обработки сообщений: 0\% -- 5, 25\% -- 11, 50\% -- 98, 75\% --
314, 99\% -- 496. Равномерный закон распределения задан с параметрами
(1,2) и выбран генератор №1. Закон распределения Пуассона задан с параметром равным 1 и выбран генератор №1. Общее число заявок — 500.

\imgw{160mm}{0.png}{Результат работы программы (вероятность повторной обработки сообщений - 0\%)}
\imgw{160mm}{25.png}{Результат работы программы (вероятность повторной обработки сообщений - 25\%)}
\imgw{160mm}{50.png}{Результат работы программы (вероятность повторной обработки сообщений - 50\%)}
\imgw{160mm}{75.png}{Результат работы программы (вероятность повторной обработки сообщений - 75\%)}
\imgw{160mm}{99.png}{Результат работы программы (вероятность повторной обработки сообщений - 99\%)}





