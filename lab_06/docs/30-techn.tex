\chapter{Технологическая часть}
В данном разделе будут представлены требования к программному обеспечению, средства реализации, листинги кода и тесты.

\section{Требования к программному обеспечению}

Вход: количество городов (целое положительноне число); симметричная матрица смежности, задающая граф; параметры для муравьиного алгоритма --- $\alpha$ (коэффициент жадности, вещественное число от 0 до 1), $p$ (коэффициент испарения, вещественное число от 0 до 1) и $t_max$ (время жизни колонии муравьев, целое положительное число).

Выход: порядок посещенных вершин в кратчайшем пути и его длина для каждого из реализованных алгоритмов, таблица параметризации для муравьиного алгоритма.

\section{Выбор средств реализации}

В качестве языка программирования для реализации данной лабораторной работы был выбран язык программирования Python  \cite{PythonBook}. Данный язык программирования позволяет реализовать муравьиный алгоритм и имеет необходимые библиотеки для формирования таблиц.

В качестве среды разработки был выбран PyCharm Professional \cite{pycharm}. Данная среда разработки является кросс-платформенной, предоставляет функциональный отладчик, средства для рефакторинга кода и возможность установки необходимых библиотек при необходимости.

\section{Реализация алгоритмов}
В листингах \ref{full} -- \ref{full-gen} представлена реализация алгоритма полного перебора,
в листинге \ref{ant} --- реализация муравьиного алгоритма, а в листинге \ref{len} --- реализация алгоритма подсчета длины пути.

\captionsetup{singlelinecheck = false, justification=raggedright}
\begin{lstlisting}[caption=Реализация алгоритма полного перебора, 
    label={full}]
def full_search(matr, n_cities):
    min_way_length = INF
    min_way = None
    cur_way = []

    while next_way(cur_way, n_cities):
        cur_way_lenth = count_way_lenth(matr, cur_way)
        if cur_way_lenth < min_way_length:
            min_way_length = cur_way_lenth
            min_way = cur_way

    return min_way_length, min_way
}
\end{lstlisting}

\begin{lstlisting}[caption=Реализация алгоритма полного перебора (генератор маршрутов), 
    label={full-gen}]
def next_way(a, n):
    if not a:
        for i in range(n):
            a.append(i)
        return True
    j = n - 2
    while j != -1 and a[j] > a[j + 1]:
        j -= 1
    if j == -1:
        return False
    k = n - 1
    while a[j] > a[k]:
        k -= 1
    a[j], a[k] = a[k], a[j]
    l = j + 1
    r = n - 1
    while l < r:
        a[l], a[r] = a[r], a[l]
        l += 1
        r -= 1
    return True
}
\end{lstlisting}

\begin{lstlisting}[caption=Реализация муравьиного алгоритма, 
    label={ant}]
def ant_search(D, n_cities, alpha=ALPHA, pho=PHO, tmax=TMAX):
    Q = 0
    for i in range(n_cities):
        for j in range(i):
            if D[i][j] < INF:
                Q += D[i][j]
    beta = 1 - alpha

    eta = [[0 for i in range(n_cities)] for j in range(n_cities)]
    tau = [[0 for i in range(n_cities)] for j in range(n_cities)]
    for i in range(n_cities):
        for j in range(i):
            eta[i][j] = 1 / D[i][j]
            eta[j][i] = 1 / D[j][i]
            tau[i][j] = 2 * EPS
            tau[j][i] = 2 * EPS

    min_way_length = INF

    for t in range(tmax):
        vis_cities = [[i] for i in range(n_cities)]
        for k in range(n_cities):
            while len(vis_cities[k]) != n_cities:
                P_ch = [0 for i in range(n_cities)]
                for j in range(n_cities):
                    if j not in vis_cities[k]:
                        i = vis_cities[k][-1]
                        P_ch[j] = (tau[i][j] ** alpha) * (eta[i][j] ** beta)
                P_zn = sum(P_ch)
                for j in range(n_cities):
                    P_ch[j] /= P_zn
                    
                probability = random()
                summ, j = 0, 0
                while summ < probability:
                    summ += P_ch[j]
                    j += 1
                vis_cities[k].append(j - 1)
            way_length = count_way_lenth(D, vis_cities[k])
            
            if way_length < min_way_length:
                min_way_length = way_length
                min_way = vis_cities[k]

        for i in range(n_cities):
            for j in range(i):
                delta_tau = 0
                for k in range(n_cities):
                    way_length = count_way_lenth(D, vis_cities[k])
                    for m in range(1, len(vis_cities[k])):
                        if (vis_cities[k][m], vis_cities[k][m - 1]) in ((i, j), (j, i)):
                            delta_tau += Q / way_length
                            break
                            
                tau[i][j] = tau[i][j] * (1 - pho) + delta_tau
                if tau[i][j] < EPS:
                    tau[i][j] = EPS
                tau[j][i] = tau[i][j]
                
    return min_way_length, min_way
\end{lstlisting}

\begin{lstlisting}[caption=Реализация алгоритма подсчета длины пути, 
    label={len}]
def count_way_lenth(D, vis_cities):
    length = 0

    for l in range(1, len(vis_cities)):
        i = vis_cities[l - 1]
        j = vis_cities[l]
        length += D[i][j]

    return length
\end{lstlisting}


\section{Тестирование}

В таблице \ref{tabular:test} приведены функциональные тесты для алгоритма полного перебора и муравьиного алгоритма. Все тесты были пройдены успешно.

\begin{table}[h!]
	\begin{center}
		
		\caption{\label{tabular:test} Тестирование реализаций алгоритмов решения задачи коммивояжера}
		\begin{tabular}{c@{\hspace{7mm}}c@{\hspace{7mm}}c@{\hspace{7mm}}c@{\hspace{7mm}}c@{\hspace{7mm}}}
			\hline
			Матрица смежности & Ожидаемый результат & Результат программы\\ \hline
			\vspace{4mm}
			$\begin{pmatrix}
				0 &  3 &  4 &  7\\
				3 &  0 &  3 &  7\\
				4 &  3 &  0 &  7\\
				7 &  7 &  7 &  0
			\end{pmatrix}$ &
			13, [0, 1, 2, 3] &
                13, [0, 1, 2, 3] \\
			\vspace{2mm}
			\vspace{2mm}
			$\begin{pmatrix}
				0 &  1 &  1 &  1 &  1 &  1\\
				1 &  0 &  1 &  1 &  1 &  1\\
				1 &  1 &  0 &  1 &  1 &  1\\
				1 &  1 &  1 &  0 &  1 &  1\\
				1 &  1 &  1 &  1 &  0 &  1\\
				1 &  1 &  1 &  1 &  1 &  0\\
			\end{pmatrix}$ &
			5, [0, 1, 2, 3, 4, 5] & 
                5, [0, 1, 2, 3, 4, 5]\\
			\vspace{2mm}
			\vspace{2mm}
			$\begin{pmatrix}
				0 & 15 & 19 & 20 \\
			15 &  0 & 12 & 13 \\
			19 & 12 &  0 & 17 \\
			20 & 13 & 17 &  0
			\end{pmatrix}$ &
			44, [0, 1, 2, 3] &
                44, [0, 1, 2, 3]\\
		\end{tabular}
	\end{center}
\end{table}

\captionsetup{singlelinecheck = false, justification=centering}

\section*{Вывод}
В данном разделе были представлены требования к программному обеспечению и средства реализации, реализованы и протестированы алгоритмы решения задачи коммивояжера.

